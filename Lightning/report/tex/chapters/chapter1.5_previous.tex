\chapter{Previous Work}
The field of distributed computing is a well developed field in computer
science. One of the most well known systems is the Berkeley Open Infrastructure
for Grid Network Computing (BOINC)\footnote{http://boinc.berkeley.edu/} which 
was designed for usage in SETI@Home project.



\section{BOINC}
luls många flops.

Differences with BOINC and the proposed project:
\begin{itemize}
	\item In BOINC, the scientists are the end user. They give the BOINC network
		a problem to solve, and they get the results. In the proposed project, 
		the user acts like both worker and job initiator. It’s more of a 
		give-and-take philosophy.
	\item The project focus more on time-dependent computations.
	\item Both projects have some kind of “credits” system, in our case these
		“credits” will play a bigger role in how much each client can utilize
		the service.
	\item The project has more control over what each hardware component does
		(ie. specific tasks for each of the GPU and CPU)
	\item Different network structure/topology. Communication in BOINC is done
		by sending data/tasks by the HTTP protocol, through centralized data
		servers that handles traffic between all connected clients which means 
		that there is no direct interaction between two clients. 
\end{itemize}

