\chapter{Previous Work}
The field of distributed computing is a well developed field in computer
science. One of the most well known systems is the Berkeley Open Infrastructure
for Grid Network Computing (BOINC)\footnote{http://boinc.berkeley.edu/}.
While BOINC is a more general approach to distributed computing other 
implementations aimed at computer graphics and rendering exists, such as Backburner, Mental Ray and SquidNet.


\section{BOINC}
Initially developed by Berkeley for the SETI@home project, BOINC is now used in
a wide range of projects, currently operating worldwide at 5,581.790 
TeraFLOPS\footnote{http://boincstats.com/stats/project\_graph.php?pr=bo 
fetched 2011-05-11}.

%Differences with BOINC and the proposed project:
The main difference between BOINC and Lightning is the network structure and topology.
Communication in BOINC is done by sending data with the HTTP protocol through centralized data
servers that handles traffic between all connected clients, which means 
that there is no direct interaction between two clients. Lightning only
requires a tracker to convey information between P2P connections.
BOINC utilizes a 'credits' system to enable friendly competition within the user base.
A similar system was initially planned for Lightning that, depending on the amount of processing power the user had previously
made available to the network, dictated how much a client could utilize the service. The idea was to have more of a give-and-take philosophy, influenced by the BitTorrent protocol.

% omskrivet till löpande text ovan:
%\begin{itemize}
%	\item In BOINC, the scientists are the end user. They provide the BOINC network
%		a problem to solve, and they get the results. In Lightning, 
%		the user acts like both worker and job initiator. It’s more of a 
%		give-and-take philosophy.
%	\item Lightning focuses more on time-dependent computations.
%	\item Both projects have some kind of ''credits'' system, but in Lightning
%		''credits'' plays a bigger role in how much each client can utilize the service.
%	\item Different network structure and topology. Communication in BOINC is done
%		by sending data with the HTTP protocol through centralized data
%		servers that handles traffic between all connected clients, which means 
%		that there is no direct interaction between two clients. Lightning only
%		requires a tracker to convey information between clients.
%\end{itemize}

\section{Backburner}
% http://images.autodesk.com/adsk/files/backburner20100.2_user_guide.pdf

Autodesk Backburner is a job management system for distributed rendering and works with several products of the Autodesk suite, such as 3D Studio Max and Maya. The backburner architecture is composed of mainly two components, Backburner Manager and Backburner Server. A user with a creative application that supports the Backburner interface sends jobs to the Backburner Manager, which distributes these jobs as a set of tasks to its connected Backburner servers. The manager keeps track of the network topology and stores a database of the current job states in its servers. A Backburner server is used together with adapters and processing engines. What type of job a server is capable of executing depends on what type of adapters and processing engines is installed on the server. Lightning is also designed with this functionality in mind, enabling the system to support virtually any processing job. However, as with most distributed rendering solutions, Backburner relies on a dedicated network / render farm, which not only requires many resources as well as network administration. With this in mind, even though Backburner is a more reliable solution, the cost is might overshadow the gain for a small home network, or a user with limited resources. 

\section{Mental Ray}

%http://www.kxcad.net/autodesk/3ds_max/Autodesk_3ds_Max_9_Reference/distributed_bucket_rendering_rollout_mental_ray_renderer.html

Mental Ray is equipped with distributed rendering capabilities similar to Lightning: a frame can be divided into buckets and rendered on machines elsewhere using Autodesk Backburner. However, Mental Ray requires the user to know the intended hosts on which the server software is running. This requires a whole lot of network administration and is best suited for dedicated networks. 

\section{SquidNet}

%http://www.squidnetsoftware.com/Documents/SquidNet-Users-Guide.pdf
%TODO:  life för mycket meningar börjar med The här

SquidNet is a distributed render system implemented as a plugin for most modern 3D modeling applications. The architecture is composed by a management console and a set of individual network engines. The management system is where a user creates jobs and submits them to a centralized network queue. In an automated process, the clients in the network (also called Tipnodes) monitor the work queue for active jobs and tag a slice of the job that they will process. Every node in the network have access to the queue and can see which job slices that are currently being processed. Tipnodes can act as either job creator, job processor or both. A processing node will do nothing but wait for jobs to spawn in the work queue, while a job creator simply sends its processing task to the management console. Nodes in the network work independent of each other; if a node stops responding, the rendering will still be done by other nodes in the network. Lightning currently has no fail-safe system such as this and requires a new network topology to be built in the tracker every time a client crashes or a user wants to abort a rendering. 






