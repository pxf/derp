\chapter{Previous Work}
The field of distributed computing is a well developed field in computer
science. One of the most well known systems is the Berkeley Open Infrastructure
for Grid Network Computing (BOINC)\footnote{http://boinc.berkeley.edu/}.
	



\section{BOINC}
Initially developed by Berkeley for the SETI@home project, BOINC is now used in
a wide range of projects, currently operating worldwide at 5,581.790 
TeraFLOPS\footnote{http://boincstats.com/stats/project\_graph.php?pr=bo 
fetched 2011-05-11}.

Differences with BOINC and the proposed project:
\begin{itemize}
	\item In BOINC, the scientists are the end user. They provide the BOINC network
		a problem to solve, and they get the results. In the proposed project, 
		the user acts like both worker and job initiator. It’s more of a 
		give-and-take philosophy.
	\item The project focus more on time-dependent computations.
	\item Both projects have some kind of “credits” system, in our case these
		“credits” will play a bigger role in how much each client can utilize
		the service.
	\item The project has more control over what each hardware component does
		(ie. specific tasks for each of the GPU and CPU)
	\item Different network structure/topology. Communication in BOINC is done
		by sending data/tasks by the HTTP protocol, through centralized data
		servers that handles traffic between all connected clients which means 
		that there is no direct interaction between two clients. 
\end{itemize}

\section{Backburner}
% http://images.autodesk.com/adsk/files/backburner20100.2_user_guide.pdf

Autodesk Backburner is a job management system for distributed rendering and works with several products of the Autodesk suite, such as 3D Studio Max and Maya. The backburner architechture is comprised of mainly two components, Backburner Manager and Backburner Server. A user with a creative application that supports the Backburner interface sends jobs to the Backburner Manager, which distributes these jobs as a set of tasks to its connected Backburner servers. The manager keeps track of the network topology and stores a database of the current job states in its servers. A Backburner server is used together with adapters and processing engines. What type of job a server is capable of executing depends on what type of adapters and processing engines is installed on the server. 

Backburner is quite similar to Lightning in several ways. 

similarities:
- Interface solution
- Different processing engines
- Batch rendering / group rendering

differences: 
- Network administration
- Render farm access
- Backburner is controlled with awesome monitor system to pause/stop/otherwise control jobs
- A dedicated network is more reliable than ad-hoc
- Backburner can schedule clients to only work on certain hours/dates

\section{Mental Ray}

%http://www.kxcad.net/autodesk/3ds_max/Autodesk_3ds_Max_9_Reference/distributed_bucket_rendering_rollout_mental_ray_renderer.html

Mental Ray is equipped with distributed rendering capabilities similar to Lightning: a frame can be divided into buckets and rendered on machines elsewhere using Autodesk Backburner. However, Mental Ray requires the user to know the intended hosts on which the server software is running. This requires a whole lot of network administration, and is best suited for dedicated render farms, something that a hobbyist or student often does not have access to. 

\section{SquidNet}

%http://www.squidnetsoftware.com/Documents/SquidNet-Users-Guide.pdf
%TODO:  life för mycket meningar börjar med The här

SquidNet is a distributed render system implemented as a plugin for most modern 3D modeling applications. The architechture is composed by a management console and a set of individual network engines. The management system is where a user creates jobs and submit them to a centralized network queue. In an automated process, the clients in the network (also called Tipnodes) monitor the work queue for active jobs and tags a slice of the job that it will process. Every node in the network have access to the queue and can see which job slices that are currently being processed. Tipnodes can act as either job creator, job processor or both. A processing node will wait for jobs to spawn in the work queue, a job 





