\chapter{Description}
%Dealing with large quantities of computational data is an ever current topic, not only in Computer Science but in various other scientific %areas aswell. Even though new and faster microprocessors reduce computational cost linearly, parallelism (where applicable) over large %problem sets yields even higher speedup gains. 

% - Speed-sensitivity
% - Old problem but new technology / techniques (p2p)
% - Combination of interesting areas?
% - Generic approach, but with specific implementation
% - Academic study of relationship between network distribution cost vs. running on one computer?  

\emph{Lightning} is a solution for distributing computational data over a peer-to-peer (P2P) network, implemented in two parts: a naive ray tracer for 3D scene rendering and communication between remote clients in a decentralized network. The two parts have mainly been developed asynchronously and are ultimately combined in a simple prototype. 

The network protocol focuses mainly on fast data transfer rather than handling
large amounts of data and will primarily be used to synchronize clients in the
network to collectively share the computational workload of a rendering task.

The simple ray tracer subdivides a 3D scene into smaller independent jobs
(screen regions / blocks). Each job is automatically distributed over the
network to the clients who wait for a job to appear on its communication
channel. After processing each job, the clients sends the finished result back
to the job initiator. The rendered frame is thus built block by block,
displaying the result to the user incrementally. 

The general purpose of this project is to examine if distributed ray tracing calculations, over a P2P network, are able to reduce the render time of an arbitrarily complex 3D scene, compared to a regular local ray tracing solution. Implementing a new ray tracing application gives a deeper knowledge on ray tracing as a topic but also a greater freedom when incorporating it into the rest of the project, compared to existing ray tracing solutions.

The project name \emph{Lightning} is a play on the word \emph{lighting}, a commonly used word in computer graphics rendering, at the same time also suggesting 'lightning fast' rendering speeds. %Along with the fact that the distributed network can be viewed as a "cloud", the name seek to establish an impression of a lightning fast rendering solution

