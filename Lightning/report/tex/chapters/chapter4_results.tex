\chapter{Results}
We have done some basic performance testing, shown in the table below.

\begin{center}
    \begin{tabular}{ | l | l | l |} \hline
    Task count & Processor cores & Total render time \\ \hline
    2x2 & 1 & 3m 30s \\ \hline
    16x16 & 14 & 8-12s \\ \hline
    32x32 & 14 & 1m 40s \\ \hline
    \end{tabular}
    % caption?
\end{center}

We can clearly see that the rendering time is spread out and varies wildly. There are
some known problems with our task distribution algorithm that can cause an uneven distribution
of jobs across all nodes. For our test case, optimal distribution was achieved with region of size 16x16, yielding a render time of only a couple of seconds.

Increasing the region size to 32x32 quadruples the number of jobs, and shows how severe the performance hit is
when the distribution algorithm woes. 

% problems with task distribution, cause of "spread-out" results
% 

\chapter{Future Work}
\section{Ray tracing algorithm}
The ray tracing and shading algorithms implemented are very naive and simple due to the time constraints in the project. If more time was given then a natural development would be to revamp the ray tracing and shading algorithms so they reach results comparable with commercial software. Another perhaps even better and more profitable approach would be to use existing ray tracing and shading solutions. This in turn would decouple the shading and clients even more, and the software would instead be viewed as a middleware between 3D modeling and shading software.
%Improved ray tracing algorithm. [More primitives, light types, blah, bleh.]

\section{SIMD Optimization}
By using SIMD (Single Instruction Multiple Data), multiple computations can be performed in parallel. If used correctly, a speedup of up to 4 times per execution thread can be expected. For us, SIMD can be applied to both ray calculations and data structures.

\section{Better ways of distributing tasks}
- cleaver way of splitting up and passing along tasks/batches\\

\section{Photon Mapping}
Distributed photon mapping would be a possible addition to the ray tracer, in order to get a good approximation of global illumination. The photon mapper would be implemented as a job processor, where each job could be divided into tasks such that each task would shoot a fix amount of photons. These results would then be collected at the job initiator and incorporated into the data used later on for the raytracing job processors.

\section{Load balancing}
To further increase performance, we would need a better task distribution algorithm that accounts for CPU count and CPU load for each available node, that possible also makes decisions based on network throughput between lightning clients and the host computer.

\section{Job initiator as plugin for commercial software}
Thanks to the decoupling of job initiators and job processors/clients, it would be possible to implement a job initiator as a plugin for a commercial modeling software, ie. 3ds Max. By doing this, the usage would fit better with the current workflow of such programs, and thus more trivial for the already established community of each product/software.

%- too much overhead?\\
%- GOOD IDEA, but only prototype. Has potential like a massive BOSS.