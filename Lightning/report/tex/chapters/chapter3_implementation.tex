\chapter{Implementation}

\section{Worker Client}
\label{sec:worker}
The client uses two bound sockets for incoming connections. The first socket is
used by the tracker after the client have initiated contact to fully establish
the connection and make sure the client is connectable. The second is used by
the job initiator and other clients when forwarding tasks that needs to be
processed. Tasks are processed by a pool of worker threads, where the size of
the pool equals the number of CPU cores in the system. Results from the
processing of a job is sent back to the job initiator by a special purpose 
\textit{Sender-}thread.

Messages passed between clients containing task data consists of initiation data
along with actual task data. In the example of ray tracing, initiation data is
the scene which is to be rendered, and task data is screen coordinates defining
what portion of the scene a task is to render. It is most often the case that
the scene data is of much larger size than the coordinates that define a task.
To avoid unnecessary retransmission of initiation data, this information is
stored in a cache even though there are no tasks present which is associated
with the data. Only after a specified amount of time has passed since a task was
processed using the data will it be removed.

The client is designed in a stateless manner, meaning that when a message is
sent to a recipient, no record is held of what has transpired in detail, except
for a simple data structure holding some information helping in decision making
by the client.

% Discussion: Att det här suger?
To make sure the network is responsive enough, every client sends PING-messages 
at a given interval, and closes the connections that does not respond in time.

\subsection{Queues}
The client has three queues, handling all the incoming, outgoing and completed
tasks.
\begin{description}
	\item[Internal queue] \hfill \\
	Contains all the tasks that the internal workers are to process. These tasks
	are reserved for the client and cannot be forwarded. The internal queue will
	stay filled if there are tasks in the external queue.
	\item[External queue] \hfill \\
	All the tasks that the client have been given, which currently does not 
	fit in the internal queue.
	\item[Outgoing queue] \hfill \\
	When a task is completed its result is put in the outgoing queue from where
	the Sender-thread dequeues it and sends the data to the job initiator.
\end{description}

\subsection{Task Distribution}
Clients start out connected to nothing but the tracker. When a connecting client
or job initiator forwards a set of tasks to the client a suitable number of
forwarding clients is calculated. A fixed setting entered at compile time represents
the preferred amount of tasks each client should receive. To prohibit that one
client forwards a too big set of tasks, thus producing a ''flat'' forwarding
distribution, another fixed setting is introduced that caps the number of
clients that can be forwarded to. If one client were to send sets of tasks to a
large amount of clients, transmitting the initiation data would add network
latency to the total time of the given job, which could be avoided to some
extent if the task of transmitting said data would also be distributed among the
clients.

When requesting new clients from the tracker
the following is performed: \\

\begin{samepage}
\begin{algorithmic}
\STATE \COMMENT{num\_tasks equals the number of tasks to be forwarded}
\IF {$num\_tasks \geq preferred\_tasks*max\_clients$}
	\STATE $request\_nodes(max\_clients$)
\ELSE
	\STATE $request\_nodes(num\_tasks/preferred\_tasks$)
\ENDIF
\end{algorithmic}
\end{samepage}

The \textit{internal} queue is always kept full as long as there is tasks in the
\textit{external} queue. This means that the client will keep reducing the tasks
awaiting processing and forwarding at all times.

It is expected that the tracker provides a set of worker clients for which to
forward the tasks to. The tasks left in the external queue are evenly
distributed among the nodes received by the tracker, leaving the external queue
empty. When the worker threads have emptied the internal queue a message to the
tracker is sent informing it of the clients availability to accept new tasks.

If the tracker can't provide idle workers, the client enters a state in which it
waits for a message from the tracker to provide a set of new worker clients. When a
message from the tracker is received informing the client about available workers,
all remaining tasks are distributed evenly among those clients received by the 
tracker, leaving the external queue empty. This means that in a saturated
network where all of the connected clients are busy at most of the times, the
chunk of tasks remaining in a job will be passed around when clients become
available. However, since most clients already have the initiation data, this
will not induce a significant network latency, since task data tend to be small.


\section{Ray-Tracer Job Processor}
Even if the worker client would be able to handle many different job processors, the main focus were from the start to create a naïve ray-tracer processor. Implementation started with simple tests consisting of ray-sphere intersections, and were further developed to handle ray-triangle intersections based on parsed scene data. The lightning calculations are also very simple, but fit the purpose of a simple prototype.

Each job is essentially a serialized 3D scene, which consists of a collection of triangles, a kd-tree (explained in more detail below), materials, light sources and the camera position and orientation. While the job initiator serialize and packages this data, it is up to the job processor to unpack the data and use it for each task.

Each of the tasks in a job, is simply a predefined region of the final image. The job processor runs a shading function for each pixel it has to render, where it essentially traces a ray from the camera origin through each pixel into the scene. The shading function uses the data sent with the job (triangles, kd-tree, material and light information) to shade the pixels in the task region. As soon as a task is finished, the results are repacked and pushed onto the outgoing queue, where the client then takes care of sending it back to the job initiator.
%- Naïve raytracer implementation\\

\subsection{KD-Tree}
In the first iteration of the ray tracer, a brute force algorithm was used for each ray against all the geometry in the scene. At first, it did not cause extreme performance costs as small scenes with simple objects was used. However, when using complex 3D objects with thousands of triangles, rendering the scene took 10+ minutes and was thus impossible to work with. To address this issue, a spatial division scheme was implemented. There exists several different schemes with various benefits, but ultimately a KD tree algorithm was designed for the task, based on a paper by Ingo Wald\footnote{http://www.sci.utah.edu/~wald/Publications/2006///NlogN/download//kdtree.pdf}. 

As a tree can take several minutes to build, it is calculated offline once and then stored locally as a binary file. When loading a model, the renderer loads the cached file containing the binary kd-tree in almost no time.

The tree is constructed offline in a number of passes:
\begin{itemize}
  \item \textbf{Pass 1:} Construct a bounding volume for each primitive used in the scene. A bounding volume that encapsulates all the primitives in the tree is also calculated. This is used for early hit detection by a ray-box intersection on the total volume.
  \item \textbf{Pass 2:} Recursively build the tree. The following pseudo-algorithm describes the algorithm:


\begin{enumerate}
	\item \texttt{if stopping criteria met then }
	\item \texttt{  create leaf node and assign it with current work data}
	\item \texttt{calculate a bounding box for left and right child nodes}
	\item \texttt{for each axis do}
	\item \texttt{ for each split position do}
	\item \texttt{   calculate total SAH cost}
	\item \texttt{   if total cost is < current best cost then}
	\item \texttt{     set best axis and split position}
	\item \texttt{if the cost of not splitting is better than splitting then}
	\item \texttt{  create leaf node and assign it with current work data}
	\item \texttt{create two inner nodes and assign with data}
	\item \texttt{recurse on left and right nodes}
\end{enumerate}


  Comments: 

  1-2: The stopping criteria in this case is met when either the work set or the recursion depth is less than specific constants. Tweaking the value of these constants yield varying intersection performance as they determine the size of the tree (and thus also the speed of the algorithm). In this version, the criteria is work size < 8 and depth < 20, which works quite well. 

  5: SAH, or surface area heuristic calculates a cost value based on the volume of the left and right bounding boxes. A minimal cost creates a tight box around the primitives in the working data set and indicates the best split position of the split candidates (the min and max points of the primitives). The current split position splits the current node's bounding volume in two sides, left and right. The total cost is calculated as follows: 

  left SAH = volume of left bounding box

  right SAH = volume of right bounding box

  left primitives and right primitives indicates the amount of primitives in either bounding volume.

  \begin{equation}
  	\text{total cost} = \text{SAH}^{\text{left}} \times \text{primitives}^{\text{left}} + \text{SAH}^{\text{right}} \times \text{primitives}^{\text{right}}
  \end{equation}

  \item \textbf{Pass 3:} Build a cache-friendly tree that uses data indices instead of raw data from the tree in pass 2. As no pointers are used in the leaves, this tree can finally be serialized to a file, or sent over a network connection and used directly by clients with no re-calculation or processing.
\end{itemize}

When the tree is build, it can then be used in ray-primitive intersections when ray tracing.

\section{Tracker}
When a node connects to the tracker, a new connection is established to the node's incoming socket from the tracker. Iif it succeeds, the first socket is closed in favor of the now outgoing connection, which provides the network with an automated way of testing whether the node is connectable or not. After the connection has been fully established, the node is given an unique ID that it uses whenever it communicates with the tracker and other nodes.
The following types of messages are then used:

\begin{description}
\item[Node requests more nodes] \hfill \\\\
Any available nodes are returned. If no such nodes exist, the requesting node is instead put in a waiting queue.
\item[Node signaling it is available to other nodes] \hfill \\\\
The node is tagged as being available for work, if there are nodes in the waiting queue already they receive this node until it's spent.
\end{description}

To be able to make smart decisions about the network topology, the tracker keeps a database on how the nodes are connected.

\section{Job Initiator}
Initial plans were to create a job initiator as a plugin to an existing 3D modeling application. These plans had to be put on hold as focus needed to be shifted to a simpler model viewer type of job initiator. The model viewer is able to load one or more 3D models. It has the possibility to render the 3D models in realtime using OpenGL, so that the user can reposition the camera before sending the scene to the network.
\pic[0.35]{img/lightning_screens.png}{\emph{\textbf{Top left:}} The job initiator showing (rendered using OpenGL locally) a simple teapot model before being it ray traced. \emph{\textbf{Top right:}} The view right after a job has been sent to the clients (no results has been received yet). Each task can be seen here with corresponding task id for easier debugging. \emph{\textbf{Bottom left:}} About half (33 of 64) of the tasks has been completed and received, as shown by the progress bar. Visual task id numbering has been disabled here. \emph{\textbf{Bottom right:}} Finished image, all tasks completed and received. Showing a simple reflection of a tea pot in a sphere.}
