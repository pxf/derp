
\chapter{Engine}
\section{Introduction}
This project uses a custom made game engine with focus on modularity and portability. The engine relies heavily on open source libraries, and is built in such a way that adding additional libraries is trivial.

The engine has a proper module system, and requires advanced components to be implemented as dynamically linked libraries. By moving all implementation details out of the core engine, the main source tree is kept clean and more easily maintainable. Inheritly because of the module system, components can easily be replaced with other implementations (e.g. support for the Xbox 360 could be provided
by creating components using the Xbox Developer Kit).

The resource manager within the engine also uses modules, which makes it possible to add support for new resource formats without replacing an entire executable. It can even
be done "on-demand" during runtime, i.e. "only load the ogg decoder if an ogg-file is requested".

 \section{Features}
 \begin{itemize}
 \item Audio
 \item Graphics (using OpenGL)
 \item Input (keyboard and mouse)
 \item Network
 \item Resource loading
 	\begin{itemize}
 		\item Audio (ogg)
 		\item Image (bmp, png, jpg, tga, dds, psd, hdr)
 		\item Mesh (ctm)
 		\item Data Interchange (json)
 	\end{itemize}
 \end{itemize}

We also use a couple of cross-platform libraries that aren't fully integrated with the game engine yet, giving us features such as
\begin{itemize}
\item Scripting with Lua, angelscript or scheme
\item Input for USB Gamepad devices such as the Xbox 360 controller.
\item Physics using Box2D
\end{itemize}

\section{Supported platforms}
Currently supported platforms include
\begin{itemize}
\item Microsoft Windows
\item Mac OS X
\item Linux (other unices are untested)
\end{itemize}
Future work will add support for iOS and possibly Android.

\section{Network}
It's always considered good practice to use already existing libraries and solutions for common problems, and in this case we were in need of fast and reliable network transmissions.
We found ENet which looked very promising, and of course it was open source. It sends data using UDP packets and have built in support for sequencing.
The network is divided into four subsystems; a network device, a client, a server and packets.
\subsection{ENet}
\subsection{Server}
\subsection{Client}
\subsection{Packet}

