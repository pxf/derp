\chapter{Discussion}

\section{Engine}

[motorn blev bäst]

\section{Network}

[efnet suger? skulle kanske valt TCP istället ja.. det är ju viktigare med reliability än hastighet etc]

\section{Renderer}

[inte så mycket att säga här kanske..? den gör ju det den ska liksom]

\section{Editor}

[snygg och fin, lätt att knåda med lua, widgetbasen kan enkelt användas till andra projekt]

[synd att vi inte hann med hueshiftgrejen, den är lite väl brun nu kanske]

\section{Future Releases}
[saker som vi inte hann med eller så]

In large projects when the workspace gets cluttered from wires and boxes, a user can batch components together in groups. These groups can then be minimized or collapsed in order to achieve a better overview of the project. This functionality also serves as a performance increase, as each component and wire require processing power to visualize. 

Originally, it was intended to include user custom color schemes, manipulated by a hue-shift shader. Different color schemes enable users to customize the look of the whole editor to better suit the users preferences, such as light-dark contrast as well as general visual profile.

[shader inspector typ? så att man kan kolla i själva knåden vad som sker, alt. ändra, eller så ser man vart kompileringen sket sig, eftersom det sker på renderaren.]

[editor resizing]

[snyggare wires]

[någonting med att man kunde koppla om inuti själva komponenten, men jag minns inte vad det var bra för..]

\chapter{Conclusion}

[här kan man slänga in lite text från project notes också, som man kan referera till.. ]

[typ nått sånt här: "When rendering the output, the backend profiles each component and sends this data along with the finished frame back to the editor. The timing information is displayed somewhere on the component itself in the editor window. "]

[på det stora hela blev ju projektet nästan exakt som vi tänkte oss. skulle behöva göra lite fler komponenter egentligen, vi har ju inga tuffa grejer att visa liksom..]