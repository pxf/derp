\chapter{Discussion}

\section{Engine}
We are quite satisfied with the game engine as a whole. So far we haven't encountered any major show stoppers, only minor issues which easily have been resolved. The module system works great as well, and the components we have implemented seem to be reasonably stable.

We were able to get the engine up and running quite quickly, many thanks to the following open source libraries: \texttt{enet}, \texttt{glew}, \texttt{glfw}, \texttt{jsoncpp}, \texttt{lua}, \texttt{openctm}, \texttt{rtaudio}, \texttt{soil}, \texttt{stb\_vorbis} and \texttt{zthread}.

\section{Network}
We initially decided to use ENet as a networking library for it's ease of use, speed and reability features. However, ENet uses UDP for communication, and it had some quirks to it when wanting to deliver packets both reliable and in-order. For this project in particular, using regular TCP sockets would have been a better choice and would have resulted in better networking \mbox{performance}. 
Since all the data we sent were important it felt like we had to re-implement all the safety TCP offered.

The fact that ENet handles all the incoming and outgoing connections in the background while waiting for new packets to arrive together with everything being based on UDP and not TCP makes it impossible to have any kind of control regarding which clients really are connected and what packets were lost, and why.

The speed that ENet offered were nothing but a disappointment, it could take up to 30 seconds to send a small picture with a local server and client without a reason.

As it turned out, it felt like using ENet was worse than using regular sockets with TCP and implementing serializing of data from scratch.


\section{Renderer}
In the planning stage of the development it was unclear how much work would have to be done on the renderer. Mainly depending on not knowing how the structure of the pipelines, and what type of components there was going to be. But as soon as we had specified the JSON format for the pipelines, it gave us an idea how we could start implementing the renderer. It was quite evident that we didn't need many different component types, but just a few that were generic enough.

In comparison to the editor, the renderer turned out very simple, it does what it should but nothing more. It also lack any type of security of the incoming data, which should be a high priority for any future work.
%[inte så mycket att säga här kanske..? den gör ju det den ska liksom]

\section{Editor}
We are surprised how well the editor turned out, and how closely it came to our original proposed idea. Even though at the beginning it felt like a monumental task, in hindsight it feels that we made the right decision to implement our own GUI system, since it gave us the freedom and flexibility we needed. It made it possible for us to optimize the UI rendering for our own needs, while still being able to rework and redo fundamental architecture inside the script host application, without worrying about underlying script functionality as long as we expose the same features to the scripts with each iteration.

One of the problems we were afraid of from the start, were if implementing all the editors functionality via scripts could result in unresponsiveness or a "sluggish feeling" on slow computers. But this didn't surface as any practical problem in the finished application at all, we were in fact able to run the editor on very low end netbook without any problem.

The one thing we feel should have addressed early in development of the GUI system, is the possibility of arbitrary window sizes and resizing. This was something we felt the need for quite late in the development process, and was something we kept pushing aside for other more urgent fixes. But implementing such a feature would be something needed for future work, and should be possible to accomplish with minor tweaks to the GUI scripts. Another feature we originally planned were theme customization, but something we had to skip implementing to save time. In the very end of the development process we came up with a compromise, to let the user change the overall hue/colors of the theme. This could be done by implementing a graphics shader that at runtime modified the colors of the theme image. But also this feature was cut due to time constraints. 

%[snygg och fin, lätt att knåda med lua, widgetbasen kan enkelt användas till andra projekt]

%[synd att vi inte hann med hueshiftgrejen, den är lite väl brun nu kanske]

\section{Future Releases}
%[saker som vi inte hann med eller så]

In large projects when the workspace gets cluttered from wires and boxes, a user can batch components together in groups. These groups can then be minimized or collapsed in order to achieve a better overview of the project. This functionality also serves as a performance increase, as each component and wire require processing power to visualize. 

Originally, it was intended to include user custom color schemes, manipulated by a hue-shift shader. Different color schemes enable users to customize the look of the whole editor to better suit the users preferences, such as light-dark contrast as well as general visual profile.

Since component output is resolved with shader programs that are compiled in the renderer, any error during compilation is only visible in the terminal window of the server. Thus, when using a remote server, a user will not receive any feedback on what line of code caused the error in the shader program, and will just receive faulty results. This is solved by sending compile information back to the user, displaying it in the editor's console window. 

An idea we had in the start of the project was to be able to view the source code for each component directly in the editor and possibly even edit it directly. However, this feature would require a huge amount of time, and would not add invaluable functionality in relation to development time. 

%[editor resizing, har vi skrivit om i discussion tror jaggg]

As this is a project in interaction design, a big part of the program is the actual visual design of the application interface. As visual sugar, we initially wanted to model the connections between components as physical wires. For instance, the cables would be affected by laws of gravity, when a user moves the plugs around, the wire would bend and wobble, just as a real one would. Similar ideas are used in Propellerhead's Reason\footnote{http://www.propellerheads.se/products/reason/}. This would be possible to implement in a future version using a third party 2D physics library, but something we didn't have time to integrate ourselves.

%[någonting med att man kunde koppla om inuti själva komponenten, men jag minns inte vad det var bra för..]

%[generic shader component]
Right now, the components yield static functionality as they are hard-coded in the Lua scripts. It would be beneficial to include a generic shader component that loads custom shader code that the user provides. The generic component yields output as any other component, and can be connected to other nodes in the graph. This gives an experienced easy access to test out whatever specially designed shader he wants. The component would have attribute name and type as inputs, and either color, depth or normal as output.

%[fler components]
Currently, only a small set of components are implemented, not near enough to build any type of visual effect as we proclaimed in the introduction of this paper. The current state of the application is basically a proof-of-concept. Now that we know our ideas are working, we could easily add more interesting components, in order to cover more techniques and algorithms.

\chapter{Conclusion}
%[blev bra, precis som vi tänk typ]
We think that the final prototype version reached the goals we aimed for in most areas.

[blev den useful? nja, inte med nuvarande komponenter?]

[jämfört med andra liknande?]

