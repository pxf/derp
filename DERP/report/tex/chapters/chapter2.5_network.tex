\section{Network}
It's always considered good practice to use existing libraries and solutions for common problems, which in this case ment fast and reliable transmission over a network.
For this purpose, the open-source library ENet was successfully integrated into the game engine. ENet sends data using UDP packets and have built in support for sequencing and connection management. 

The network is divided into four subsystems; a network device, a client, a server and packets.

A network device is used to manipulate and create new instances of servers and clients. The separation of the two is to not hinder implementation of new features such as several servers in the same application (e.g. the renderer might want to have several editors connected at the same time).

ENet supports arbitrary amount of data channels. To use these it is required to create tags (i.e. named channels) before spawning servers and clients. The different tags enables easy management for arbitrary types of data.

\begin{description}
	\item[A server] binds to a port and incoming connections gets handled while \mbox{receiving} or sending packets. There are several methods available to unicast or multicast packets to connected clients.
	\item[A client] connects to a server and functions in a similiar way as a server, however it is only possible to connect to one server at a time thus only unicasting packets are available.
\end{description}

A packet consists of a string identifier, and arbitrary amount of smaller \mbox{objects} such as integers, strings etc. that are stored as binary data blobs.
Identification of packets can be used by to connect sent packets with received packets in an easy way without having to send extra unnecessary objects along with the more important data, saving both bandwidth and time for the developer.
When a packet has been populated with data by using the method \texttt{PushObject(..)} and are to be sent over a connection, the data blobs are serialized along with the identification specified in the creation phase.


