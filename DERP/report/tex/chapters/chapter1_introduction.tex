\chapter{Introduction}

Developing advanced graphical applications is a very time-consuming and complex task that require experienced programmers with a strong knowledge in computer science and engineering. However, with an increase of hobbyists and enthusiasts in the field of computer graphics during the recent years, developing intuitive development tools is always motivated. A visual artist or a novice programmer with a passion for video games might lose interest due to the steep learning curve of 3D software development.

Effects such as blinn/phong-lighting, shadow mapping, bloom filters etc. are very common and frequently implemented in various 3D applications and games. While shader code usually is easy to reuse, it often requires a programmer who knows the intended framework by heart. An artist who quickly wants to see their models/textures/leveldesign/etc. with in-game lighting, or with whatever effects the engine supports, might face difficulties because of the extra overhead caused by not knowing low-level API's or the graphics engine. Nowadays, most game engines provide users with deeply integrated visual tools to overcome such issues. These tools are thus application specific, and often restricted in functionality. 
 
\section{Description}

The DERP Editor is a content creation system that enables a user to build materials and post-processing rendering effects for 3D applications. The system is composed of several independent modules in a plug-in based architecture. The DERP editor is a small, standalone, easy-to-use piece of software with an intuitive graphical user interface, targeted towards anyone interested in computer graphics. It could be used by anyone starting out in computer graphics for the first time, or by a visual artist with no programming background or as a means of rapid prototyping. 

\section{Motivation}
%[Det här känns lite fattigt..]
While similar tools can be found within various other modern 3D engines and packages, such as Autodesk Maya 3D or the Unreal Engine 3 to name a few, most such applications are proprietary, bound to other products and often quite expensive. The aim of this project is to extend and further develop these ideas, providing a computer graphics enthusiast with all the necessary tools to build and test whatever techniques he wishes in an intuitive environment. 
 
From an academic point of view, this project is an excellent opportunity in order to gain valuable experience in developing unique / interesting ideas in a contemporary environment. 
 