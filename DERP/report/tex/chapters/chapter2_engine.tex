
\chapter{Engine}
\section {Introduction}
This project uses a custom made engine for interactive real-time applications (essentially a \textit{game engine}) with focus on modularity and portability. The game engine relies heavily on non-restricted open source libraries, and is built in such a way that adding additional third party libraries is trivial.

Our motivation for writing a game engine from scratch is driven by both the opportunity to learn from
constructing such a system, and to have a solid and portable code-base on which to build future products.

\section {Build system}
A major problem when developing software for multiple target systems is a unified build system that work
identically for all target systems. This engine uses a fast and flexible build system named \textbf{bam}\footnote{http://matricks.github.com/bam/}, and
with this tool we have constructed a build script that automatically resolves library dependencies for each supported platform. Because of this, the game engine is just as easy to compile on either Microsoft Windows, GNU/Linux or Mac OS X. Bam can use multiple threads to decrese compilation times, and it can easily be used with Visual Studio for development on Windows.

Our build script is \textit{module aware}, and it supports compiling modules directly into the engine if needed. A built-in module can ofcourse be overridden by another dynamic library. The beauty of this is that the developer can decide for himself how he want to package the resulting product.

The build script provide primitives to easily get started with new projects. 

\newpage

\section{Module system}
The engine has a proper module system, and requires complex components to be implemented as separate dynamically linked libraries. By moving all implementation details out of the core engine, the main source tree is kept clean and is more easily maintainable. Inheritly because of the module system, components can easily be replaced with other implementations (e.g. support for the Xbox 360 could be provided
by creating components using the Xbox Developer Kit).

\section{Resource management}
Resource management are devided into two parts; resource loaders and resources. For example, for loading images
there is a factory class \texttt{ImageLoader}, which in a couple of different way can produce instances of a class \texttt{Image}.

The following resource loaders are currently available, and adding support for new resource types is trivial.


\begin{tabular}{l l}
\textbf{Type}    & \textbf{supported file formats}\\
\hline
Text    & txt  (any text data)                  \\
Blob    & blob (any binary data)                \\
Audio   & ogg  (ogg vorbis)                     \\
Image   & bmp, png, jpeg, tga, dds, psd, hdr    \\
Font    & pfnt (custom font format)             \\
Mesh    & ctm  (compressed triangle mesh)       \\
Data Interchange & json                         \\
\end{tabular}


The resource manager within the engine also uses the module system, which makes it possible to add support for new resource formats without the need to create a new executable with built-in support. Resource loaders can even be added \textit{on-demand} during runtime, i.e. only load the ogg audio decoder \textit{if} an ogg audio-file is requested.

\newpage

\section{Input}
The engine ships with a rudimentary input system supporting both keyboard and mouse. Nothing exciting or
innovative going on here yet. The API currently support two ways for polling keyboard and mouse status:
\begin{description}
\item[Check current status] Check if a \textit{specified} key/button is currently being held down. Return a boolean value depending on key status.
\item[Check cache] Get the latest key/button that has been pressed and released. Return key/button that was pressed. The cache has to be explicitly cleared.
\end{description}

The first alternative works best for interactive use, i.e. \textit{move forward if} \texttt{W} \textit{is being held down}, while the alternative cache version works better for checking general keypresses. I.e. \textit{Was a key pressed? Yes, }\texttt{ESC} \textit{was pressed. Cancelling current action.}
 
\section{Graphics}
The graphics component of the game engine is built with portability in mind, and by that we mean portability
between different render APIs. Currently there is only support for OpenGL, but it should be possible to implement the same functionality using Direct3D for example.

The OpenGL implementation is far from complete, but it support the most commonly used functionality such as textures, vertex buffer objects, GLSL shaders and framebuffers.


In addition to a basic renderer abstraction, the engine includes utilities for quickly drawing different types of primitives, support for bitmapfonts with kerning, loading textures from a variety of formats and loading meshes into vertex buffers.

\section{Audio}
[skriv bättre, flytta lite till future extensions]
The game engine currently implements a very basic audio mixer. Playing sounds is very simple, and works fine
for various sound effects (e.g \textit{click}-sound for mouse clicks) or music. Up to 16 voices can be played at once. Architecture for audio effects, or any kind of filters is currently missing (3D audio is not possible).

\section{Future extensions}

We also use a couple of cross-platform libraries that aren't fully integrated with the game engine yet, giving us features such as
\begin{itemize}
\item Scripting with Lua, angelscript or scheme
\item Input for USB Gamepad devices such as the Xbox 360 controller.
\item Physics using Box2D
\end{itemize}

\section{Supported platforms}
Currently supported platforms include
\begin{itemize}
\item Microsoft Windows
\item Mac OS X
\item Linux (other unices are untested)
\end{itemize}
Future work will add support for iOS and possibly Android.

